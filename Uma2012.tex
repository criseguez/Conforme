%% LyX 2.0.0 created this file.  For more info, see http://www.lyx.org/.
%% Do not edit unless you really know what you are doing.
\documentclass[11pt,spanish]{article}
\usepackage[T1]{fontenc}
\usepackage[latin1]{inputenc}
\usepackage{amsmath}
\usepackage{amssymb}

\makeatletter
%%%%%%%%%%%%%%%%%%%%%%%%%%%%%% User specified LaTeX commands.


\usepackage{latexsym}

\topmargin 0mm
\oddsidemargin 5mm
\evensidemargin 5mm
\textwidth 150mm
\textheight 662.801 pt



\makeatother

\usepackage{babel}
\addto\shorthandsspanish{\spanishdeactivate{~<>}}

\begin{document}
\begin{center}
\textsf{~\\[14pt] %%%%%% Insertar el t�tulo de la contribuci�n %%%%%%%
Soluciones a problemas de membranas de geometr�as diversas sometidas
a condiciones el�sticas sobre la frontera y a distribuciones generales
de cargas} 
\par\end{center}

%%%%%% Nombres de los autores y sus respectivas filiaciones: %%%%%%


{\footnotesize { }{\footnotesize \par}

\begin{center}
{\footnotesize Antonio S�ngari$^{a}$ y Cristina Eg�ez$^{b}$ \\[14pt]}
\par\end{center}{\footnotesize \par}

\begin{center}
{\footnotesize $^{a}$Universidad Nacional de Salta - Consejo de Investigaci�n
de UNSa \\[3mm]}
\par\end{center}{\footnotesize \par}

\begin{center}
{\footnotesize $^{b}$Universidad Nacional de Salta - Consejo de Investigaci�n
de UNSa \\[3mm]}
\par\end{center}{\footnotesize \par}

{\footnotesize }}{\footnotesize \par}

%%%%%%%  Insertar el resumen %%%%%%
 %%%%%%%%%  Por favor, no usar s�mbolos o fuentes no estandard.
Un problema de inter�s en ingenier�a trata sobre encontrar soluciones
al problema de membranas vibrantes. Bajo ciertas condiciones, la ecuaci�n
diferencial que modeliza las membranas es invariante frente a transformaciones
conformes. Entonces, el problema puede reducirse a la transformaci�n
de dominios mediante algoritmos, como el de Koebe, que transforma
regiones generales simplemente conexas en el c�rculo unitario. Como
segunda etapa se puede transformar este c�rculo en un rect�ngulo,
tambi�n mediante transformaciones conformes. En este trabajo se expone
el m�todo y se dan resultados num�ricos

%%%%%%%%%%%%%%%%%%%%%%%%%%%%%%%%
%   Bibliograf�a
%%%%%%%%%%%%%%%%%%%%%%%%%%%%%%%%




{\small %%%%% Ejemplo de formato:
}{\small \par}
\begin{thebibliography}{Referencias}
{\small \bibitem{PS}Papamichael y Stanopulos.}\end{thebibliography}

\end{document}
