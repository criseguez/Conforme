
\documentclass[a4paper,spanish]{article}
%%%%%%%%%%%%%%%%%%%%%%%%%%%%%%%%%%%%%%%%%%%%%%%%%%%%%%%%%%%%%%%%%%%%%%%%%%%%%%%%%%%%%%%%%%%%%%%%%%%%%%%%%%%%%%%%%%%%%%%%%%%%%%%%%%%%%%%%%%%%%%%%%%%%%%%%%%%%%%%%%%%%%%%%%%%%%%%%%%%%%%%%%%%%%%%%%%%%%%%%%%%%%%%%%%%%%%%%%%%%%%%%%%%%%%%%%%%%%%%%%%%%%%%%%%%%
\usepackage{babel}

%TCIDATA{OutputFilter=LATEX.DLL}
%TCIDATA{Version=5.50.0.2953}
%TCIDATA{<META NAME="SaveForMode" CONTENT="1">}
%TCIDATA{BibliographyScheme=Manual}
%TCIDATA{Created=Monday, September 10, 2007 11:32:14}
%TCIDATA{LastRevised=Thursday, May 17, 2012 08:03:24}
%TCIDATA{<META NAME="GraphicsSave" CONTENT="32">}
%TCIDATA{<META NAME="DocumentShell" CONTENT="Standard LaTeX\Blank - Standard LaTeX Article">}
%TCIDATA{CSTFile=LaTeX article.cst}
%TCIDATA{PageSetup=0,0,0,0,0}
%TCIDATA{Counters=alph,3}
%TCIDATA{AllPages=
%H=0
%F=0
%}


\newtheorem{theorem}{Theorem}
\newtheorem{acknowledgement}[theorem]{Acknowledgement}
\newtheorem{algorithm}[theorem]{Algorithm}
\newtheorem{axiom}[theorem]{Axiom}
\newtheorem{case}[theorem]{Case}
\newtheorem{claim}[theorem]{Claim}
\newtheorem{conclusion}[theorem]{Conclusion}
\newtheorem{condition}[theorem]{Condition}
\newtheorem{conjecture}[theorem]{Conjecture}
\newtheorem{corollary}[theorem]{Corollary}
\newtheorem{criterion}[theorem]{Criterion}
\newtheorem{definition}[theorem]{Definition}
\newtheorem{example}[theorem]{Example}
\newtheorem{exercise}[theorem]{Exercise}
\newtheorem{lemma}[theorem]{Lemma}
\newtheorem{notation}[theorem]{Notation}
\newtheorem{problem}[theorem]{Problem}
\newtheorem{proposition}[theorem]{Proposition}
\newtheorem{remark}[theorem]{Remark}
\newtheorem{solution}[theorem]{Solution}
\newtheorem{summary}[theorem]{Summary}
\newenvironment{proof}[1][Proof]{\noindent\textbf{#1.} }{\ \rule{0.5em}{0.5em}}
\topmargin=-3.5cm
\oddsidemargin=0cm
\textwidth=16.5cm
\textheight=27.6cm
\pagestyle{empty}
\input{tcilatex}
\begin{document}

\title{\textbf{Implementaci\'{o}n de una transformaci\'{o}n asociada a un
homeomorfismo entre una regi\'{o}n plana simplemente conexa y el disco
unitario abierto}\\
Geometr\'{\i}a}
\author{Antonio S\'{a}ngari, Cristina Eg\"{u}ez \\
%EndAName
Departamento de Matem\'{a}tica, Facultad de Ciencias Exactas. \\
Universidad Nacional de Salta}
\maketitle

Construimos una transformaci\'{o}n biyectiva conforme que convierte regiones 
$\Omega $ simplemente convexas del plano (distintas del mismo plano) en el c%
\'{\i}rculo unitario abierto $U$, y rec\'{\i}procamente. En este trabajo
consideramos dominios incluidos en el c\'{\i}rculo unitario que contienen el
origen y cuya frontera corta solamente una vez los rayos desde el origen.
Para ello usamos el esquema de la demostraci\'{o}n del Teorema de Riemann
para transformaciones conformes, realizada por Koebe, que consiste en la
construcci\'{o}n iterativa de regiones $\Omega _{1},$$\Omega _{2},\dots $, a
partir de la regi\'{o}n $\Omega _{0}=\Omega \subset U$, generadas por una
sucesi\'{o}n de funciones $f_{1},f_{2},\dots $, de modo que $f_{i}\left(
\Omega _{i-1}\right) =\Omega _{i}$, y $f_{n}\circ f_{n-1}\circ \cdots \circ
f_{1}$ converja a una transformaci\'{o}n conforme $f$ de $\Omega $ sobre $U$%
. A su vez cada una de las funciones $f_{i}$ es la composici\'{o}n de dos
transformaciones bilineales y una raiz cuadrada. La demostraci\'{o}n este
teorema hace uso del hecho que toda funci\'{o}n anal\'{\i}tica que no se
anula en $\Omega $ admite una funci\'{o}n ra\'{\i}z cuadrada tambi\'{e}n an%
\'{a}litica en $\Omega $.

Planteamos una rotaci\'{o}n al semiplano de la derecha que nos permite
aplicar luego una raiz cuadrada anal\'{\i}tica, y lograr as\'{\i} un
homeomorfismo entre cualquier regi\'{o}n propia del plano, simplemente
conexa, y el disco unitario abierto, como obsevamos en la figura 1 para
dominios convexos y en la figura 2 para dominios c\'{o}ncavos.%
\[
\FRAME{itbpFU}{3.5172in}{0.9219in}{0in}{\Qcb{{\protect\small Figura 1}}}{}{%
figura1.jpg}{\special{language "Scientific Word";type
"GRAPHIC";maintain-aspect-ratio TRUE;display "USEDEF";valid_file "F";width
3.5172in;height 0.9219in;depth 0in;original-width 3.896in;original-height
0.9997in;cropleft "0";croptop "1";cropright "1";cropbottom "0";filename
'figura1.jpg';file-properties "XNPEU";}}
\]%
\[
\FRAME{itbpFU}{3.4869in}{0.8916in}{0in}{\Qcb{{\protect\small Figura 2}}}{}{%
figura2.jpg}{\special{language "Scientific Word";type
"GRAPHIC";maintain-aspect-ratio TRUE;display "USEDEF";valid_file "F";width
3.4869in;height 0.8916in;depth 0in;original-width 4.0309in;original-height
1.0101in;cropleft "0";croptop "1";cropright "1";cropbottom "0";filename
'figura2.jpg';file-properties "XNPEU";}}
\]

Elaboramos un paquete inform\'{a}tico que construye cada una de las $f_{i}$,
las aplica a la frontera de $\Omega _{i-1}$ y obtiene la frontera de $\Omega
_{i}$. Una vez concluida esta tarea, usando la transformaci\'{o}n inversa,
que tambi\'{e}n ser\'{a} conforme, podemos conseguir una transformaci\'{o}n
de $U$ sobre $\Omega $.

\begin{thebibliography}{9}
\bibitem{1} EG\"{U}EZ C., S\'{A}NGARI A., Aplicaci\'{o}n del algoritmo de
Koebe a regiones simplemente conexas, Comunicaciones Cient\'{\i}ficas. UMA
2010.

\bibitem{2} HILBERT D. and COHN-VOSEN S., Geometry and the Imagination,
Chelsea Publishing Company New York, 1990.

\bibitem{3} NEVANLINNA R., PAATERO V, Introduction to complex analysis (AW,
1969)

\bibitem{4} RUDIN W., Real and Complex Analysis, McGraw-Hill, 1992.
\end{thebibliography}

\end{document}
