\documentclass[11pt,spanish]{article}

\usepackage{latexsym}
\usepackage{amssymb,amsmath}
\usepackage[latin1]{inputenc}
\usepackage[spanish]{babel}

\topmargin 0mm
\oddsidemargin 5mm
\evensidemargin 5mm
\textwidth 150mm
\textheight 662.801 pt

\begin{document}

\begin{center}
{\sf ~\\[14pt]
 %%%%%% Insertar el t�tulo de la contribuci�n %%%%%%%
Soluciones a problemas de membranas de geometr�as diversas sometidas
a condiciones el�sticas sobre la frontera y a distribuciones generales
de cargas }
 \end{center}

 %%%%%% Nombres de los autores y sus respectivas filiaciones: %%%%%%

 \footnotesize{
   \begin{center}
     Antonio S�ngari$^a$, y Cristina Eg�ez$^b$ \\[14pt]

     $^a$Universidad Nacional de Salta - Consejo de Investigaci�n
de UNSa   \\[3mm]

     $^b$Universidad Nacional de Salta - Consejo de Investigaci�n
de UNSa  \\[3mm]
   \end{center}
   }



 \normalsize
 \noindent
 %%%%%%%  Insertar el resumen %%%%%%
 %%%%%%%%%  Por favor, no usar s�mbolos o fuentes no estandard.
Una cuesti�n de inter�s en ingenier�a es la b�squeda de soluciones al problema de la membrana vibrante. Bajo ciertas condiciones, la ecuaci�n
diferencial que modeliza las membranas es invariante frente a transformaciones
conformes. Entonces, el problema puede reducirse a la transformaci�n
de dominios mediante algoritmos, como el de Koebe, que transforma
regiones generales simplemente conexas en el c�rculo unitario. Como
segunda etapa se puede transformar este c�rculo en un rect�ngulo,
tambi�n mediante transformaciones conformes. En este trabajo se expone
el m�todo y se dan resultados num�ricos


%%%%%%%%%%%%%%%%%%%%%%%%%%%%%%%%
%   Bibliograf�a
%%%%%%%%%%%%%%%%%%%%%%%%%%%%%%%%

{\small \begin{thebibliography}{99}

%%%%% Ejemplo de formato:

\bibitem{PS}N. Papamichael y N. Stylianopoulos, Numerical Conformal Mapping, World Scientific Publishing Co. Pte. Ltd.
\ {\bf } (2010).

\bibitem{H} P. Henrici, Applied and Computacional Complex Analysis. Vol 3, John Wiley & Son, Inc. (1986).

\bibitem{DT} T. Driscoll y L. Trefethen, Schwarz-Christoffel Mapping , Cambridge Unirversity Press Son, Inc. (2003)


\end{thebibliography}}

\end{document}
